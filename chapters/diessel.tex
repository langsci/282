\documentclass[output=paper,colorlinks,citecolor=brown]{langscibook} 
\author{Holger Diessel\affiliation{Friedrich Schiller University Jena}\orcid{}\lastand Merlijn Breunesse\affiliation{University of Amsterdam}\orcid{}}
\title{A typology of demonstrative clause linkers}
\abstract{Across languages, demonstratives provide a frequent diachronic source for a wide range of grammatical markers including certain types of clause linkers such as English \textit{so}, \textit{that}, \textit{thus} and \textit{therefore}. Drawing on data from a sample of 100 languages, this chapter presents a cross-linguistic survey of (grammaticalised) demonstratives that are routinely used to combine clauses or propositions. The study shows that demonstrative clause linkers occur in a large variety of constructions including all major types of subordinate clauses and paratactic sentences. Concentrating on the most frequent types, the chapter considers (grammaticalised) demonstratives functioning as (i) relative pronouns, (ii) linking and nominalising articles, (iii) quotative markers, (iv) complementisers, (v) conjunctive adverbs, (vi) adverbial subordinate conjunctions, (vii) correlatives and (viii) topic markers. It is the purpose of the chapter to provide a comprehensive overview of demonstrative clause linkers from a cross-linguistic perspective and to consider the mechanisms of change that are involved in the grammaticalisation of demonstratives in clause linkage constructions.}
\IfFileExists{../localcommands.tex}{
  % add all extra packages you need to load to this file

\usepackage{tabularx,multicol}
\usepackage{url}
\urlstyle{same}

\usepackage{listings}
\lstset{basicstyle=\ttfamily,tabsize=2,breaklines=true}

\usepackage{tabularx}
\usepackage{langsci-optional}
\usepackage{langsci-lgr}
\usepackage{langsci-gb4e}

\usepackage[linguistics,edges]{forest}

\usepackage{tikz}





  \newcommand*{\orcid}{}

\makeatletter
\let\thetitle\@title
\let\theauthor\@author
\makeatother

\newcommand{\togglepaper}[1][0]{
  \bibliography{../localbibliography}
  \papernote{\scriptsize\normalfont
    \theauthor.
    \thetitle.
    To appear in:
    Change Volume Editor \& in localcommands.tex
    Change volume title in localcommands.tex
    Berlin: Language Science Press. [preliminary page numbering]
  }
  \pagenumbering{roman}
  \setcounter{chapter}{#1}
  \addtocounter{chapter}{-1}
}

\newcommand{\glopa}{\textsc{opa}}
\newcommand{\glme}{\textsc{top}}
\newcommand{\glte}{\textsc{nfin}}
\newcommand{\glta}{\textsc{nfin}}
\providecommand{\citegen}[1]{\citeauthor{#1}'s (\citeyear*{#1})}

% \newcommand{\sectref}[1]{Section~\ref{#1}} 
  %% hyphenation points for line breaks
%% Normally, automatic hyphenation in LaTeX is very good
%% If a word is mis-hyphenated, add it to this file
%%
%% add information to TeX file before \begin{document} with:
%% %% hyphenation points for line breaks
%% Normally, automatic hyphenation in LaTeX is very good
%% If a word is mis-hyphenated, add it to this file
%%
%% add information to TeX file before \begin{document} with:
%% %% hyphenation points for line breaks
%% Normally, automatic hyphenation in LaTeX is very good
%% If a word is mis-hyphenated, add it to this file
%%
%% add information to TeX file before \begin{document} with:
%% \include{localhyphenation}
\hyphenation{
affri-ca-te
affri-ca-tes
ana-phor-ic
poly-semy
Spra-chen
Fi-scher
Al-chemist
de-mon-stra-tive
Mül-ler
Brea-ker
Hix-kar-ya-na
Que-chua
Unga-rin-jin
Alam-blak
Wam-bon
da-ta-base
quasi-quo-ta-tions
Pisch-lö-ger
de-mon-stra-tions
Ar-khan-gel-skiy
}
\hyphenation{
affri-ca-te
affri-ca-tes
ana-phor-ic
poly-semy
Spra-chen
Fi-scher
Al-chemist
de-mon-stra-tive
Mül-ler
Brea-ker
Hix-kar-ya-na
Que-chua
Unga-rin-jin
Alam-blak
Wam-bon
da-ta-base
quasi-quo-ta-tions
Pisch-lö-ger
de-mon-stra-tions
Ar-khan-gel-skiy
}
\hyphenation{
affri-ca-te
affri-ca-tes
ana-phor-ic
poly-semy
Spra-chen
Fi-scher
Al-chemist
de-mon-stra-tive
Mül-ler
Brea-ker
Hix-kar-ya-na
Que-chua
Unga-rin-jin
Alam-blak
Wam-bon
da-ta-base
quasi-quo-ta-tions
Pisch-lö-ger
de-mon-stra-tions
Ar-khan-gel-skiy
} 
  \togglepaper[1]%%chapternumber
}{}

\begin{document}
\maketitle 

%%Still to be done:
%Orphan control
%Keep examples together
%How can page numbers be added to \citegen? Attributes in [...] do not work.
%Check formatting of tables and position of list of abbreviations

\section{Introduction}\label{sec:diessel:1}

Demonstratives are a unique class of expressions that are foundational to social interaction, discourse processing and grammar evolution \citep{Diessel2006,Diessel2013,Diessel2014}. In face-to-face conversation, demonstratives are commonly used with reference to entities in the surrounding speech situation in order to coordinate the interlocutors’ joint focus of attention. In this use, they are often accompanied by pointing gestures and other non-verbal means of deictic communication (\citealt{Bühler1934}; see also \citealt{CoventryEtAl2008}).

All languages use demonstratives for spatial reference, but demonstratives are additionally also frequently used with reference to linguistic elements in discourse \citep{HallidayHasan1976}. Two basic discourse uses are commonly distinguished: the tracking use, in which demonstratives refer to discourse participants, and the discourse-deictic use, in which demonstratives refer to an adjacent clause or proposition \citep{Webber1991}.

In addition to these uses, many languages have grammatical function morphemes that are historically derived from demonstratives. In the grammaticalisation literature, it is often assumed that all grammatical function morphemes are ultimately based on content words \citep[111]{HeineKuteva2007}, but, as \citet{Brugmann1904} and \citet{Bühler1934} noted, demonstratives also provide a frequent source for the development of grammatical markers. There are a wide range of grammatical function words that are frequently derived from tracking and discourse deictic demonstratives, including definite articles, third person pronouns, relative pronouns, copulas and subordinate conjunctions (\citealt{Himmelmann1997,Diessel1999Book}, \citeyear{Diessel1999Article}). Some of these markers have been studied intensively from both diachronic and cross-linguistic perspectives. There is, for instance, a great deal of research on the development of definite articles from tracking or anaphoric demonstratives in a large number of languages (e.g. \citealt{Harris1978,Cyr1993,Laury1997}). However, other types of development have not been systematically investigated from a cross-linguistic perspective. Conjunctive adverbs, for instance, are frequently based on discourse-deictic demonstratives, but there is almost no research on this topic \citep[125-127]{Diessel1999Book}.

In this chapter, we will be concerned with (grammaticalised) demonstratives that are routinely used for clause linkage. In English, for example, the expressions \textit{so}, \textit{that}, \textit{so that}, \textit{thus} and \textit{therefore} are based on demonstratives and serve to combine clauses or propositions. Similar types of grammaticalised demonstratives occur in many other languages \citep{Himmelmann1997,HeineKuteva2007}. It is the purpose of this chapter to show that demonstratives are of central significance to the development of grammatical markers in the domain of clause linkage. More specifically, the chapter provides a typology of “demonstrative clause linkers” and analyses the mechanisms of change behind their development. 

Since demonstratives are commonly used with reference to linguistic elements in the unfolding discourse, they provide a natural starting point for the grammaticalisation of clause linkers \citep{Bühler1934,Diessel2012}. Yet, while the frequent development of demonstratives into clause linking morphemes is motivated by discourse-pragmatic factors, it is important to recognise that the grammaticalisation of clause linkers is also influenced by the syntactic properties of demonstratives in particular constructions (\citealt{Himmelmann1997,Diessel1999Book}, \citeyear{Diessel1999Article}).

Early research on grammaticalisation has focused on semantic and pragmatic changes of lexical expressions, but more recent research has shown that grammaticalisation processes involve constructions \citep{Traugott2003}, or entire networks of constructions (\citealt{TraugottTrousdale2013}; see also \citealt{Diessel2019Adverbials}), rather than just isolated items. Thus, in order to understand how demonstratives grammaticalise into clause linkers, one must not only consider their discourse functions but also their occurrence in particular constructions. 

The chapter builds on previous research on the grammaticalisation of demonstratives (e.g. \citealt{Brugmann1904}; \citealt{Bühler1934}; \citealt{Himmelmann1997}; \citealt{Diessel1999Book}, \citeyear{Diessel1999Article}, \citeyear{Diessel2006}, \citeyear{Diessel2014}) but is more detailed and comprehensive than all previous accounts. In fact, to the best of our knowledge, this is the first large-scale typological study that systematically investigates the role of (grammaticalised) demonstratives in the domain of clause linkage. The analysis is based on a typological database including information from a genetically and geographically dispersed sample of 100 languages. The languages come from 80 genera, with maximally two languages from each genus, and are roughly equally distributed across the six major geographical areas that are commonly distinguished in typology, i.e. Eurasia, Africa, South East Asia and Oceanic, Australia and New Guinea, South America and North America \citep{Dryer1992}. The bulk of the data have been gleaned from reference grammars and other published sources, but for some languages we also consulted native speakers and language experts. A complete list of languages included in our sample is given in the appendix.

Most of the variables in our database concern parameters of synchronic variation, but we have also gathered information on the diachronic developments of complex sentences and the various types of clause linkers. Since many clause linkers are only weakly grammaticalised, they are (often) etymologically transparent. There is plenty of evidence in our database that relative markers, complementisers, conjunctive adverbs and many other types of clause linkers are etymologically related to demonstratives. However, the mechanisms of change that are involved in the diachronic development of demonstrative clause linkers are often difficult to analyse. As we will see, in many cases we know that a particular clause linker has a deictic origin, but since there are no diachronic corpora to study the constructional changes that are involved in the grammaticalisation of demonstratives into clause linkers, we do not always know how they evolved. Nevertheless, while the source constructions of grammatical markers are frequently unknown, there is enough evidence in our database (and the historical literature) to propose some plausible scenarios of constructional change for most of the demonstrative clause linkers in our sample.

In what follows, we analyse eight different types of clause linking morphemes that are frequently derived from a demonstrative. We begin with relative pronouns (\sectref{sec:diessel:2}), which have been very prominent in the older literature on grammaticalisation (\citealt{Brugmann1904}; \citealt[402]{Bühler1934}), and then turn to a wide range of other markers, including linking and nominalising articles (\sectref{sec:diessel:3}), quotative markers (\sectref{sec:diessel:4}), complementisers (\sectref{sec:diessel:5}), conjunctive adverbs (\sectref{sec:diessel:6}), adverbial subordinate conjunctions (\sectref{sec:diessel:7}), correlatives (\sectref{sec:diessel:8}), and topic markers (\sectref{sec:diessel:9}).

\section{Relative pronouns}\label{sec:diessel:2}

The term relative pronoun is used in various ways by different scholars (see \citet[248-252]{Lehmann1984} and \citet{VanderAuwera1985} for discussion); but for the purpose of this study, we adopt the following definition: A relative pronoun is an anaphoric pronoun that represents the head noun at the beginning of a postnominal relative clause.

Relative pronouns are frequent in European languages, but rare outside of Europe \citep{Comrie2006}. In our sample, there are four European languages in which relative pronouns descended from a question word (French, Georgian, Hungarian, Serbo-Croatian) and one language (German), in which relative pronouns are based on a demonstrative.\footnote{In Hungarian, relative pronouns are derived from question words by the prefix \textit{a-}, which is historically related to the demonstrative \textit{az} ‘that’, e.g. \textit{a-ki} ‘\textsc{that-}who’ \citep[40]{KeneseiEtAl1998}.} As can be seen in \REF{ex:diessel:1a}-\REF{ex:diessel:1c}, in German relative clauses are introduced by a demonstrative relative pronoun that indicates the syntactic function of the head noun through case-marking or a preposed adposition.

\ea\label{ex:diessel:1}
{Modern German (Indo-European, Germanic)}\\
\ea\label{ex:diessel:1a}
\gll   Das  ist   der   Mann,  \textbf{der}   mir   geholfen   hat.\\
       this  is   the   man  \textsc{rel.nom} me  helped   has\\
\glt ‘This is the man who helped me.’
\ex\label{ex:diessel:1b}
\gll   Das  ist   der   Mann,  \textbf{den}  ich   gesehen   habe.\\
       this  is   the   man  \textsc{rel.acc}  I  saw  have\\
\glt ‘This is the man who I saw.’
\ex\label{ex:diessel:1c}
\gll   Das  ist der   Mann,  \textbf{mit}  \textbf{dem}  ich gesprochen   habe.\\
       this  is the   man  with  \textsc{rel.dat}   I spoken  have\\
\glt ‘This is the man who I talked to.’
\z
\z

Outside of Europe, there are only four other languages in our sample in which relative pronouns are introduced by a demonstrative that qualifies as a relative pronoun according to our definition. One of them is Tümpisa Shoshone \REF{ex:diessel:2}.

\ea\label{ex:diessel:2}
\langinfo{Tümpisa Shoshone}{Uto-Aztecan, Numic}{\citealt[358, 359]{Dayley1989}}\\
\ea\label{ex:diessel:2a} 
\gll   Wa’ippü    nia  pusikwa   {\ob}\textbf{atü}   hupiatüki-tü{\cb}.\\
       woman.\textsc{sbj}  me  know  {\db}that.\textsc{sbj}   sing-\textsc{ptcp.prs.ss.sbj}\\
\glt ‘The woman who is singing knows me.’
\ex\label{ex:diessel:2b}
\gll   Wa’ippüa   nüü  pusikwa  {\ob}\textbf{akka}   hupiatüki-tünna{\cb}.\\
       woman.\textsc{obj}  \textsc{1sg}  know {\db}that.\textsc{obj}   sing-\textsc{ptcp.prs.ss.obj}\\
\glt ‘I know the woman who’s singing.’
\z
\z

In Tümpisa Shoshone, relative clauses are commonly introduced by a case-marked demonstrative pronoun which \citet[357]{Dayley1989} classifies as a “relative pronoun”. Note, however, that while the Shoshone relative pronouns are inflected for case (and number), like those in many European languages, they do not signal the syntactic function of the head within the relative clause but agree in case (and number) with the preceding noun.

Another language in which relative clauses are introduced by demonstratives that may be analysed as relative pronouns is Yagua, an Amazonian language of Peru. There are two relative markers in Yagua \REF{ex:diessel:3} \citep[342-346]{PaynePayne1990}: (i) a “relative particle” that consists of the demonstrative \textit{jirya} and the second position clitic \textit{-tìy}, and (ii) a set of “relative pronouns” that that agree with the preceding head in class and number. Note that the “relative pronouns” are not inflected for case, but in oblique relatives, demonstratives (or third person pronouns) are combined with bound adpositions that specify the syntactic role of the head in the relative clause \REF{ex:diessel:3b}, like oblique relative pronouns in German \REF{ex:diessel:1c}.

\ea\label{ex:diessel:3}
\langinfo{Yagua}{Peba-Yaguan}{\citealt[345, 346]{PaynePayne1990}}\\
\ea\label{ex:diessel:3a}
\gll   vánu  {\ob}\textbf{jiy-ra-tìy} ray-dííy-tániy-jáy   jantya-sįį-níí{\cb}  ...\\
       man  {\db}this-\textsc{clf.n-rel} \textsc{1sg}-see-\textsc{cause-prox} imitate-\textsc{nmlz-3sg}\\
\glt ‘The man I showed the picture ...’
\ex\label{ex:diessel:3b}
\gll   sa-rą́vą́ą́   {\ob}\textbf{rá-mu-tìy}  riy-pų́ų́tya-jada  jąą́yanú-miy{\cb}  ...\\
       \textsc{3sg}-poison {\db}\textsc{inan-loc-rel}  \textsc{3pl}-paint-\textsc{pst} fer.de.lance-\textsc{pl}\\
\glt ‘His poison in which the fer-de-lances painted themselves ...’
\z
\z

Similar types of relative pronouns occur in Tamashek, a Berber language of Mali and Algeria, in which relative clauses are introduced by a demonstrative that hosts an adposition clitic if the head serves an oblique role in the relative clause \REF{ex:diessel:4a}-\REF{ex:diessel:4b}.

\ea\label{ex:diessel:4}
\langinfo{Tamashek}{Afro-Asiatic, Berber}{\citealt[633, 636]{Heath2005}}\\
\ea\label{ex:diessel:4a}
\gll   é-hæn    {\ob}\textbf{w-\'ɑ=dæɤ}  t-\`əzəbbu-ɤ{\cb}\\
       \textsc{m.sg-}house  {\db}\textsc{m-dem.sg=}in  \textsc{impf-}go.down\textsc{.impf-1sg.sbj}\\
\glt ‘The house in which I go down (= spend the night).’

\ex\label{ex:diessel:4b}
\gll   æ-h\'ɑləs {\ob}\textbf{w-\`ɑ=s}     Ø-æb\`ɑ      rure\textbf{-s}{\cb}.\\
       \textsc{m.sg-}man {\db}\textsc{m-dem.sg=ins} \textsc{3m.sg.sbj-}be.lost\textsc{.pfv}  son\textsc{-3sg.poss}\\
\glt ‘The man whose son was lost (= died).’
\z
\z

Since postnominal relative clauses including a relative pronoun are similar to paratactic sentences, it is often assumed that relative pronouns are derived from anaphoric demonstrative pronouns of structurally independent sentences that have been downgraded to subordinate clauses (\citealt[224-229]{HeineKuteva2007}; \citealt[105]{Givón2009}). The hypothesis is not implausible, but difficult to verify by concrete diachronic data \citep[282-286]{HarrisCampbell1995}. In fact, the diachronic data suggest that relative clauses typically develop under the influence of multiple source constructions \citep{Hendery2013}. For instance, \citet{Lockwood1968} argued that the relative clauses of Modern German are related to an old \textit{apo koinou} construction in which a demonstrative pronoun served a double role in main and subordinate clauses \REF{ex:diessel:5} (see also \citealt{Pittner1995}).

\ea\label{ex:diessel:5}
\langinfo{Old High German}{Indo-European, Germanic}{\citealt[243]{Lockwood1968}}\\
\gll   thô  liefun   sâr  \textbf{thie}   nan  minnôtun meist.\\
     then  ran  at.once  \textsc{dem.nom} him  loved  most\\
\glt ‘Then ran at once those who loved him most.’
\z

The sentence in \REF{ex:diessel:5} includes a demonstrative that serves as subject of two verbs: \textit{liefun} ‘ran’ and \textit{minnôtun} ‘loved’. According to \citet[242-244]{Lockwood1968}, \textit{apo koinou} constructions are easily extended to relative clauses if the two roles of the demonstrative are expressed by separate (pro)nouns (cf. \textit{Wer} \textit{ist} \textbf{\textit{die,} \textit{die}} \textit{aufgeht} \textit{aus} \textit{der} \textit{Wüste} ‘Who is the one who rises from the desert’; see also \citet[IV: 189-191]{Paul1916-1920}). Since constructions of this type were frequent in Old and Middle High German, it is not implausible that they influenced the development of relative clauses; but that does not mean that paratactic sentences did not also impact their development. As \citet{Hendery2013} has shown, relative clauses are often historically related to more than one source. In the current case, we know that relative pronouns often develop from demonstratives, but this development may involve demonstratives in several source constructions (cf. \citealt[378-383]{Lehmann1984}; \citealt[120-123]{Diessel1999Book}).

\section{Linking and nominalising articles}\label{sec:diessel:3}

Since subordinate clauses are frequently expressed by nominalisations \citep{Lehmann1988}, they are often marked by the same morphemes as noun phrases. For instance, in many languages subordinate clauses are accompanied by articles or determiners that one might analyse as particular types of clause linkers. \citet{Dryer1989} defined the term \textit{article} by two features: (i) articles are used to indicate (in)definiteness and/or (ii) serve as formal markers of noun phrases. The articles of subordinate clauses are of the latter type. They are formal markers of nominal constituents but do not indicate (in)definiteness.

Two basic types of subordinating articles may be distinguished: (i) linking articles and (ii) nominalising articles. The two types of articles from a continuum, but for the purpose of this study we reserve the term linking article for markers that are primarily used to combine a head noun with attributes, and we use the term nominalising article for markers that are primarily used to form nominal constituents. Crucially, both types of articles are commonly derived from demonstratives. In many Austronesian languages, for example, attributes are linked to the head noun by an article, as in \REF{ex:diessel:6} from Toba Batak.

\ea\label{ex:diessel:6}
\langinfo{Toba Batak} {Austronesian, Malayo-Polynesian}{\citealt[186]{Foley1980}}\\
\ea\label{ex:diessel:6a}
\gll   bijang  \textbf{na}  balga\\
       dog  \textsc{lk}  big\\
\glt   ‘a big dog’
\ex\label{ex:diessel:6b}
\gll   baoa  \textbf{na}  mang-arang  buju  i\\
       man  \textsc{lk}  \textsc{act-}write  book  the\\
\glt   ‘the man who wrote the book’
\z
\z

As can be seen, the adjective in \REF{ex:diessel:6a} and the relative clause in \REF{ex:diessel:6b} are linked to a preceding noun by the marker \textit{na}, which \citet[186-187]{Foley1980} calls a “ligature” and \citet[173]{Himmelmann1997} a “linker” or “linking article”. Similar types of linking articles occur in many other Austronesian languages, including Tagalog, Wolio and Ilokano. In all of these languages, relative clauses are linked to the preceding noun by the marker \textit{na} or \textit{a}, which is historically related to the medial demonstrative \textit{*a/na} of Proto-Austronesian (\citealt[164]{Himmelmann1997}; \citealt[100]{Ross1988}). While \textit{a/na} is also used with adjectives and other types of noun modifiers, it is particularly frequent with relative clauses \citep{Foley1980}.

Linking articles are very common in the Austronesian language family, but are also found in many other languages across the world. \citet{Schuh1983,Schuh1990} and \citet{Hetzron1995} showed that they are widely used in Chadic, Cushitic and Semitic languages, and \citet{Aristar1991} presented data from a wide range of languages in which relative clauses and genitive attributes are marked by the same linker. All of these studies emphasise that linking articles are very frequent with relative clauses and commonly derived from demonstratives.

Like linking articles, nominalising articles are often based on demonstratives. Consider, for instance, the examples in \REF{ex:diessel:7a}-\REF{ex:diessel:7c} from Chumash, in which relative clauses are syntactic nominalisations marked by the article \textit{l=} and the dependent proclitic \textit{hi=}, which, according to \citet[46]{Wash1999}, is based on a demonstrative. Since nominalised clauses serve as syntactic NPs, they can be used without a nominal head as free relatives \REF{ex:diessel:7c}.

\ea\label{ex:diessel:7}
\langinfo{Chumash}{Isolate}{\citealt[76, 97, 77]{Wash2001}}\\
\ea\label{ex:diessel:7a} 
\gll   \textbf{hi=l}=xɨp\\
       \textsc{dp=art}=rock\\
\glt   ‘a/the rock’
\ex\label{ex:diessel:7b}
\gll   hi=l=xɨp-xɨp-ʔ     \textbf{hi=l}=ʔ-iy-saʔ-išmax-šiš\\
       \textsc{dp=art-}rock-rock\textsc{-em}   \textsc{dp=art=nmlz-pl-fut-}throw.at\textsc{-recp}\\
\glt   ‘(and) rocks that they can throw at one another’
\ex\label{ex:diessel:7c}
\gll   ʔi=s-ušk̓ál    \textbf{hi=l}=ʔ-iy-qili-ʔ-aqmil.\\
       \textsc{top=3-}be.strong  \textbf{\textsc{dp-art}}\textsc{-nmlz-pl-hab-ep-}drink\\
\glt   ‘What they used to drink was strong.’
\z
\z

Similar types of nominalising articles (derived from demonstratives) occur in other languages of our sample. In Jamul Tiipay, for instance, nominal clauses and internally-headed relatives are marked by the demonstrative clitic \textit{=pu} \REF{ex:diessel:8b}-\REF{ex:diessel:8c}, which also occurs with nouns \REF{ex:diessel:8a}. Note that the demonstrative clitic is a determiner that cannot be used as an independent pronoun like the demonstratives of many other languages \citep{Diessel2005Clauses}, and that \textit{=pu} is followed by a case clitic if the subordinate clause functions as subject of the main verb \REF{ex:diessel:8c}.\footnote{Jamul Tiipay is a “marked nominative language” in which subjects are marked by a case morpheme, whereas objects are “zero-marked” \citep{Comrie2013}.}

\ea\label{ex:diessel:8}
\langinfo{Jamul Tiipay}{Hokan, Yukan}{\citealt[153, 220, 208]{Miller2001}}\\
\ea\label{ex:diessel:8a} 
\gll   wa\textbf{=pu}\\
       house=\textsc{dem}\\
\glt   ‘that house’
\ex\label{ex:diessel:8b}
\gll   {\ob}puu-ch  wi’i-x{\cb}=\textbf{pu}  uuyaaw.\\
       {\db}that.one-\textsc{sbj}  do-\textsc{irr=dem}  know\\
\glt   ‘I know she will do it.’
\ex\label{ex:diessel:8c}
\gll   {\ob}’iipa peya nye-kwe-’iny{\cb}=\textbf{pu}=ch    mespa.\\
       {\db}man this 3/1-\textsc{sjrel}-give=\textsc{dem=sbj}   die\\
\glt   ‘The man who gave me this died.’
\z
\z

Very similar types of nominal and internally headed relative clauses occur in other languages of our sample. In Assiniboine, for example, nominalised subordinate clauses are marked by the distal demonstrative \textit{žé} ‘that’ \REF{ex:diessel:9}, or, less frequently, by the proximal demonstrative \textit{né} ‘this’ \citep[415-417]{Cumberland2005}.

\ea\label{ex:diessel:9}
\langinfo{Assiniboine}{Siouan}{\citealt[347, 415, 417]{Cumberland2005}}\\
\ea\label{ex:diessel:9a}
\gll   {\ob}wį́yą  \textbf{žé}{\cb} Ø-hą́ska.\\
       {\db}woman  that  \textsc{a3}-be.tall\\
\glt   ‘That woman (over there) is tall.’
\ex\label{ex:diessel:9b}
\gll   John  {\ob}mnatkį-kte-šį    \textbf{žé}{\cb}  snok-Ø-yá.\\
       John  {\db}\textsc{Ø-a1sg}-drink-\textsc{pot-neg}  that  \textsc{st-a3}-know\\
\glt   ‘John knows that I’m not going to drink it.’
\ex\label{ex:diessel:9c}
\gll   {\ob}wįc\textsuperscript{h}ášta t\textsuperscript{h}imáni   Ø-hí    \textbf{žé}{\cb} mi-nékši  Ø-é.\\
       {\db}man   visit  \textsc{a3}-arrive.here  that   \textsc{1.poss}-uncle  \textsc{a3}.be\\
\glt   ‘The man who came to visit us is my uncle.’
\z
\z

According to \citet{Schuh1990} and \citet{Aristar1991} linking articles are often based on demonstrative pronouns that were originally used as heads of complex NPs. The best evidence for this development comes from Akkadian, an old Semitic language of Mesopotamia with extensive diachronic records \citep{Deutscher2000,Deutscher2009}. 

Like many other Afro-Asiatic languages, Akkadian had a linking article that occurred with nominal attributes. In Old and Middle Babylonian (1950 BC to 1000 BC), the linker \textit{ža} was an invariable marker, but this marker developed from the demonstrative pronoun \textit{šu}, which was inflected for gender, number and case. Analysing data from Old Akkadian (2500 BC to 1950 BC), \citet{Deutscher2001,Deutscher2009} showed that \textit{šu} was originally the pronominal head of a genitive attribute that was later extended to relative clauses. Both genitive attributes and relative clauses were frequently used with a demonstrative pronoun as head in Old Akkadian but, crucially, in the course of the development, \textit{šu} lost its status as a pronoun and turned into a formal marker of certain types of attributes. Since \textit{šu} was originally the head of a complex NP, the new genitive and relative constructions marked by \textit{šu} (or \textit{ža}) could be used without a co-occurring noun as syntactic nominalisations. Nevertheless, since the \textit{šu}-nominalisations were often used in apposition to a preceding noun, they regained their original function as noun modifiers \REF{ex:diessel:10}.

\ea\label{ex:diessel:10}
{[[\textit{šu}]\textsc{\textsubscript{prn}} [\textsc{gen} or \textsc{rc]}\textsc{\textsubscript{mod}}]\textsc{\textsubscript{np}} > [\textit{šu} \textsc{gen} or \textsc{rc}]\textsc{\textsubscript{np}} > [NP]\textsc{\textsubscript{np}} [\textit{šu} \textsc{gen} or \textsc{rc}]]\textsc{\textsubscript{np}}}\\
\z

The development of the Akkadian linker provides a plausible account for many of the properties that are characteristic of linking and nominalising articles: It explains why relative clauses are often marked by the same demonstrative linker as genitive attributes and why in many languages relative clauses can be used without a (pro)nominal head as free nominals or syntactic nominalisations (\citealt{Schuh1983}, \citeyear{Schuh1990}; \citealt{Aristar1991}).

\section{Quotative marker}\label{sec:diessel:4}

A quotative marker is a conjunction-like element that serves to mark direct speech. In some languages, quotative markers are based on general speech verbs meaning ‘say’, ‘talk’ or ‘speak’. The development of quotative markers from speech verbs has been very prominent in early research on grammaticalisation \citep{Lord1993,Klamer2000}; but, as \citet{Güldemann2008} showed, based on data from African languages, quotative markers are also frequently derived from manner demonstratives. This is confirmed cross-linguistically by our data.

Manner demonstratives are a particular subclass of demonstratives that serve to draw interlocutors’ attention onto the manner of an action \citep{König2012}. In English, manner demonstratives are complex forms consisting of the similative marker \textit{like} and a demonstrative pronoun (e.g. \textit{He did it like this}); but in many other languages, manner demonstratives are simple lexemes, which may or may not be formally related to demonstrative pronouns. In German, for instance, the manner demonstrative \textit{so} is formally distinct from demonstrative pronouns, but in Ambulas, a Sepik language of Papua New Guinea, manner demonstratives include the same deictic roots as all other demonstratives (see \tabref{tab:diessel:1}).

\begin{table}
\begin{tabularx}{\textwidth}{lXlll}
\lsptoprule
& {Pronouns} & {Determiners} & {Locative}  & {Manner}\\
\midrule
Proximal & {\textit{dé-\textbf{ké}n} ‘\textsc{3sg}-this’} & {\textit{\textbf{ké}ni} ‘this’} & {\textit{\textbf{ké}ba} ‘here’} & {\textit{\textbf{ké}ba} ‘so/thus’}\\
Distal & {\textit{dé-\textbf{wa}n} \textsc{‘3sg}-that’} & {\textit{\textbf{wa}ni} ‘that’} & 
{\textit{\textbf{wa}ba} ‘there’} & {\textit{\textbf{wa}ga} ‘so/thus’}\\
\lspbottomrule
\end{tabularx}
\caption{Demonstratives in Ambulas \citep[56-7]{Wilson1980}}
\label{tab:diessel:1}
\end{table}

Like all other demonstratives, manner demonstratives can refer to entities in the surrounding speech situation, but there seems to be a general tendency to use them with reference to sentences or propositions \citep{König2012}. In particular, manner demonstratives are often used to indicate direct speech, as in \REF{ex:diessel:11} and \REF{ex:diessel:12} from German and French.

\ea\label{ex:diessel:11}
{German (Indo-European, Germanic)}\\
\gll Ja, ich würde das \textbf{so} sagen: “Das ist ein Sonderfall.”\\
     yes I would that so say {\db}this is a special.case\\
\glt ‘Well, I would put it this way: “This is a special case.”’
\z

\ea\label{ex:diessel:12}
{French (Indo-European, Romance)}\\
\gll Marie s’est exprimée \textbf{ainsi}: “{Puisqu’il le faut}, j’irai.”\\
     Marie \textsc{refl.}is express.\textsc{ptcp} thus {\db}{since.it.must.be} I.go.\textsc{fut}\\
\glt ‘Marie expressed herself in this way: “Since it is necessary, I will go.”’
\z

Similar uses occur in many other languages of our sample. For instance, in Bariai, an Austronesian language of New Britain, the verb \textit{keo} ‘say’ is frequently accompanied by a manner demonstrative to mark direct speech \REF{ex:diessel:13}.

\ea\label{ex:diessel:13}
\langinfo{Bariai}{Austronesian, Malayo-Polynesian}{\citealt[157]{GallagherBaehr2005}}\\
\gll Taine toa   oa i-keo pa-n   \textbf{bedane}, “Gergeu  ne  taine”.\\
     female \textsc{given}   there \textsc{sbj.3sg}-say at-\textsc{3sg.obl}  like.this {\db}child  here  female\\
\glt ‘That woman spoke to him like this, “This child is a girl.”’
\z

Interestingly, some languages use different types of manner demonstratives for previous and subsequent quotations. In Ambulas, for example, \textit{kéba} ‘so/thus’ refers to a subsequent quote, whereas \textit{waga} ‘so/thus’ is referring backwards. A parallel contrast occurs in Usan, a Papuan language of New Guinea, in which \textit{ete} ‘thus’ is used to announce upcoming speech \REF{ex:diessel:14}, whereas \textit{ende} ‘so/thus’ refers to a preceding quotation (see also Korafe; \citealt{Farr1999}: 276).
%\todo[inline]{Consider referring to Teptiuk's paper in this volume.}

\ea\label{ex:diessel:14}
\langinfo{Usan}{Trans-New Guinea, Madang}{\citealt[184]{Reesink1984}}\\
\gll munon eng  \textbf{ete} yo-nob qâm-ar:  “mâni  âib  ne-teib-âm,”\\
     man   the  thus me-with say-\textsc{3sg.pst}   {\db}food  big  you-give.\textsc{sg.fut-1sg}\\
\gll \textbf{ende}  qâm-arei.\\
     thus    say-\textsc{3sg.pst}\\
\glt ‘The man said thus to me: “I will give you a lot of food,” thus he said.’
\z

There is a fluid transition between the discourse-deictic use of manner demonstratives and grammaticalised quotative markers. In the examples considered thus far, the demonstratives are only weakly grammaticalised. Yet, there are languages in which manner demonstratives have developed into true quotative markers. Meithei, for example, has “quotative complementizers” \citep[190]{Chelliah1997} that are derived from the verb \textit{háy}\textbf{ }‘say’, the nominaliser \textit{-pə} and a demonstrative clitic, i.e. \textit{=si} \textsc{prox} or \textit{=tu} \textsc{dist} \REF{ex:diessel:15}.

\ea\label{ex:diessel:15}
\langinfo{Meithei}{Sino-Tibetan, Kuki-Chin}{\citealt[305]{Chelliah1997}}\\
\gll Tomba-nə Tombi-nə má-pu ŋay-həw-li \textbf{háy-pə=du} kaw-thok-ləm-í.\\
     Tomba-\textsc{ct} Tombi-\textsc{ct} he-\textsc{pat} wait-\textsc{start-prog} say-\textsc{nmlz}=that forget-\textsc{out-evd-nhyp}\\
\glt ‘Tomba forgot that Tombi had been waiting for him.’
\z

Like Meithei, Thai Kamti has grammaticalised quotative markers that are derived from the verb \textit{waa\textsuperscript{3}} ‘say’ and the proximal demonstrative \textit{nai\textsuperscript{1}} ‘this’. The two morphemes have fused into one word that is often reduced to \textit{wan\textsuperscript{1} }in quotative constructions \REF{ex:diessel:16}.

\ea\label{ex:diessel:16}
\langinfo{Thai Kamti}{Tai-Kadai, Kam-Tai}{\citealt[123]{Inglis2014}}\\
\gll “maeu\textsuperscript{4} mai\textsuperscript{3} man\textsuperscript{4}   khaeu\textsuperscript{3} han\textsuperscript{5} uu\textsuperscript{5}”   \textbf{wan}{\textsuperscript{1}}.\\
     {\db}\textsc{2sg}   \textsc{obj}  \textsc{3sg}   want  see  \textsc{impf}  \textsc{quote}\\
\glt ‘He says that he wants to see you.’
\z

Note that the quotative marker in \REF{ex:diessel:16} is not accompanied by a speech verb. Since \textit{wan\textsuperscript{1}} includes the verb \textit{waa\textsuperscript{3}} ‘say’, one might think of \textit{wan\textsuperscript{1}} as some kind of verb, but it is not unusual that quotative markers are used without a verb. In German, for example, the manner demonstrative \textit{so} can refer to direct speech without a co-occurring verb \REF{ex:diessel:17} \citep{Golato2000}.

\ea\label{ex:diessel:17}
{German (Indo-European, Germanic)}\\
\gll Und ich  \textbf{so}:  “Okay,  das   ist  deine  Chance.”\\
     and  I  thus   {\db}okay  this  is  your   chance\\
\glt ‘And I am like, “Okay, this is your chance.”’
\z

Similar types of non-verbal quotative clauses occur in other languages of our sample, as for instance in Komnzo \REF{ex:diessel:18}, a Yam language of Papua New Guinea.

\ea\label{ex:diessel:18}
\langinfo{Komnzo}{Yam, Tonda}{\citealt[331]{Döhler2018}}\\
\gll naf   \textbf{nima}  “Nakre,   wimäs=en  mni b=ŋasog.”\\
     \textsc{3sg.erg} like.this {\db}Nakre   mango.tree=\textsc{loc}  fire \textsc{med=2:3.sbj.npst.ipfv}.climb\\
\glt ‘He (said): “Nakre! The fire is climbing up the mango tree.”’
\z

Interestingly, \citet[322-326]{Güldemann2008} argued that demonstrative quotative markers can acquire properties of verbs when they are routinely used in non-verbal clauses to mark direct speech. In Epena Pedee, for instance, the manner demonstrative \textit{má-ga} ‘that-like’ may be inflected for tense if it is not accompanied by a speech verb \REF{ex:diessel:19}.

\ea\label{ex:diessel:19}
\langinfo{Epena Pedee}{Choco}{\citealt[63, 176]{Harms1994}}\\
\ea\label{ex:diessel:19a}
\gll   \textbf{má}-ga  hara-\textbf{hí},  “…”\\
       that-like  tell-\textsc{pst}\\
\glt ‘He said as follows “…”’
\ex\label{ex:diessel:19b}
\gll   \textbf{má}-ga-\textbf{hí},  “pháta     kho-páde  a-hí.”\\
       that-like-\textsc{pst}  {\db}plantain  eat-\textsc{imp}    say-\textsc{pst}\\
\glt   ‘That is: “Eat your plantains.”’
\z\z

%repair citegen - doesn't allow for page numbers
The data from Epena Pedee provide good evidence for \citeauthor{Güldemann2008}’s (\citeyear[529]{Güldemann2008}) claim that quotative constructions provide “a highly fruitful cradle of new verbs”. 

\section{Complementisers}\label{sec:diessel:5}

In formal syntax, a complementiser is a particular word class category that serves as head of a “complementizer phrase” \citep{Radford1997}. However, in what follows, we use the term complementiser in a more traditional way for subordinate conjunctions of nominal clauses functioning as subject or object of the main verb.

Like many other types of clause linkers, complementisers are often based on demonstratives. English \textit{that} and German \textit{dass} are well-known examples. There are several other languages in our sample in which nominal clauses are marked by a demonstrative. In fact, we have already seen some examples in \sectref{sec:diessel:3}. Recall that the nominal clauses of Jamul Tiipay and Assiniboine are marked by a clause-final demonstrative \REF{ex:diessel:8}.

The position of the complementiser correlates with the order of verb and object and the position of the nominal clause relative to the main verb \citep{SchmidtkeBodeDiessel2017}. In OV languages, nominal clauses usually precede the main verb and include a clause-final complementiser, as in Jamul Tiipay and Assiniboine, whereas in VO languages, nominal clauses typically follow the main verb and are marked by an initial complementiser, as in English and German. There are several other languages with initial and final demonstrative complementisers in our sample. Consider, for instance, \REF{ex:diessel:20} and \REF{ex:diessel:21} from Amele and Tamashek.

\ea\label{ex:diessel:20}
\langinfo{Amele}{Trans-New Guinea, Madang}{\citealt[47]{Roberts1987}}\\
\gll {\ob}Naus uqa   uqa   na   ho qo-i-a   \textbf{eu}{\cb}   ija d-ug-a.\\
      {\db}Naus \textsc{3sg}   \textsc{3sg}   of   pig hit\textsc{-3sg-pst} that   \textsc{1sg} know\textsc{-1sg-pst}\\
\glt ‘I know that Naus killed his pig.’
\z

\ea\label{ex:diessel:21}
\langinfo{Tamashek}{Afro-Asiatic, Berber}{\citealt[674]{Heath2005}}\\
\gll \`ənne-ɤ=ɑ-s      {\ob}\textbf{\`ɑ}=d    i-nz̩ər{\cb}\\
     say\textsc{.pfv-1sg.sbj=dat-3sg}  {\db}\textsc{dem=com}  \textsc{3m.sg.sbj-}sing\textsc{.impf}\\
\glt ‘I told him to sing.’ (Lit. ‘I said to him, that he sing.’)
\z

Amele is an OV language in which nominal clauses precede the main clause predicate, and Tamashek is a VO language in which nominal clauses are postposed to the main verb. As can be seen, like Jamul Tiipay and Assiniboine, Amele marks preverbal nominal clauses by a clause-final demonstrative; like English and German, Tamashek marks postverbal nominal clauses by a clause-initial demonstrative. Other languages in which complementisers are based on demonstratives include Chumash, Lakhota and Diegueño. 

Note that while demonstrative complementisers are not uncommon, they are less frequent than many other types of demonstrative clause linkers in our database. In particular, the markers of relative clauses are more often based on demonstratives than the markers of nominal clauses. Concentrating on those markers for which we were able to determine a diachronic source, more than 50\% of all (free) relative markers are based on demonstratives in our data, whereas only about 15\% of all complementisers are related to demonstratives. What is more, with one exception (see below), all of the demonstrative complementisers included in our database also occur in relative clauses, suggesting that complementisers and relativisers are historically related (e.g. English \textit{that}). 

In the literature it is often said that demonstrative complementisers derive from discourse-deictic demonstratives \citep[287]{HarrisCampbell1995}. In particular, it is widely assumed that the German complementiser \textit{dass} developed from a paratactic demonstrative pronoun (\citealt[30]{Behaghel1928}; \citealt[26]{Ebert1978}). According to the standard analysis, \textit{dass} has evolved from a cataphoric demonstrative that served to anticipate an upcoming sentence as in \textit{Listen to \textbf{this}: John and Sue will get married}. On this account, the grammaticalisation of \textit{dass} involved several related changes whereby a cataphoric demonstrative pronoun turned into a formal marker of the subsequent sentence that was downgraded to a subordinate clause. This analysis is based on the occurrence of the demonstrative \textit{thaz} in two different structural positions in Middle High German \REF{ex:diessel:22}.

\ea\label{ex:diessel:22}
\langinfo{Middle High German}{Indo-European, Germanic}{\citealt[25, 26]{Axel2009}}\\
\ea\label{ex:diessel:22a}
\gll   Joh   gizálta  in  sar  \textbf{tház} {\textbackslash}   thiu   sálida  untar  ín  was.\\
       and   told  them  immediately  that  {} he   luck  among  them  was\\
\glt   ‘And told them immediately that good fortune was among them.’
\ex\label{ex:diessel:22b}
\gll   “Íh,”  quad  er,   “infúalta {\textbackslash}  \textbf{thaz}  étheswer    mih   rúarta;” …\\
       {\db}I  said  he   {\db}felt {}   that  someone  me  touched\\
\glt   ‘“I,” he said, “feel, that someone touched me;” …
\z
\z

In \REF{ex:diessel:22a} \textit{thaz} occurs at the end of the first sentence and seems to anticipate the subsequent clause, and in \REF{ex:diessel:22b} \textit{thaz} occurs at the beginning of the second sentence where it seems to serve as a formal marker of a nominal clause. Given that some authors of that period used the demonstrative \textit{thaz} in both ways (e.g. Otfried), it seems plausible to assume that the alternation between the two uses of \textit{thaz} reflects ongoing syntactic change.

However, several recent studies have questioned this view \citep{Lühr2008,Axel2009,SchmidtkeBode2014}. According to \citet{Axel2009}, there is little evidence for the cataphoric use of demonstrative pronouns in Middle High German. The few examples that are commonly cited to illustrate this use, notably \REF{ex:diessel:22a}, are unclear and leave room for alternative interpretations \citep[25]{Axel2009}. Challenging the traditional view, Axel and Lühr suggest that the complementiser \textit{dass} did not develop from a cataphoric demonstrative but from a relative pronoun. In particular, they argue that \textit{dass} emerged in the context of a correlative construction in which the relative pronoun \textit{thaz} occurred together with a demonstrative or correlative pronoun in the preceding main clause \REF{ex:diessel:23}.

\ea\label{ex:diessel:23}
\langinfo{Middle High German}{Indo-European, Germanic}{\citealt[29]{Axel2009}}\\
\gll Er   tháhta   odowila   \textbf{tház} {\textbackslash}  \textbf{thaz}  er ther  dúriwart     wás.\\
     he   thought   maybe  that  {} that  he the  doorkeeper   was\\
\glt ‘He thought that maybe he was the doorkeeper.’
\z

Correlative constructions of this type were frequent in Middle High German and provide a plausible bridging context between relative and nominal clauses. Moreover, the scenario that \citeauthor{Axel2009} and \citeauthor{Lühr2008} suggest for German is consistent with the scenario that has been proposed for other languages in which relative and nominal clauses include the same marker (\citealt{Givón1991}; \citealt[248-254]{SchmidtkeBode2014}). As pointed out above, if nominal clauses are marked by a demonstrative, relative clauses often include the same demonstrative, which is readily explained if we assume that demonstrative complementisers derive from demonstrative relativisers.

Nevertheless, there is a second scenario whereby a demonstrative pronoun may develop into a complementiser. As \citet{Lord1993} and others have shown, complementisers are frequently derived from quotative markers. Since quotative markers are often based on manner demonstratives (cf. \sectref{sec:diessel:4}), it is a plausible hypothesis that complementisers may develop from demonstratives via quotative constructions. The grammaticalisation literature has concentrated on the development of complementisers from speech verbs, but there is at least one language in our sample in which a complementiser (that does not also occur in relative clauses) may have evolved from a demonstrative quotative marker. In Noon, direct and indirect speech are marked by the “manner adverb” \textit{an} meaning ‘thus’ or ‘in this way’ \REF{ex:diessel:24a}. Since \textit{an} is also used as a complementiser with verbs of cognition \REF{ex:diessel:24b}-\REF{ex:diessel:24c}, it is not unreasonable to assume that the complementiser use of \textit{an} has developed from its use in quotative constructions.

\ea\label{ex:diessel:24}
\langinfo{Noon}{Niger-Congo, Atlantic}{\citealt[314]{Soukka2000}}\\
\ea\label{ex:diessel:24a} 
\gll   Yaal-aa    hay-ya,    woˈ-ˈa-ri     \textbf{an}:  “Mi   hot-in  ee-fu.”\\
       man-\textsc{def}    come-\textsc{narr}  say-\textsc{narr-obj.3sg}   thus   {\db}I  see-\textsc{pfv}  mother-\textsc{2sg}\\
\glt ‘The man came and said to him/her (this): “I’ve seen your mother.”’
\ex\label{ex:diessel:24b}
\gll   Ya  halaat-ee  \textbf{an}:  “Mi   hot-oo     ken.”\\
       s/he  think-\textsc{pst}  \textsc{comp}   {\db}I   see-\textsc{pres.neg}   nobody\\
\glt   ‘S/he thought (this): “I don’t see anybody.”’
\ex\label{ex:diessel:24c}
\gll   Cica     foog-ee   \textbf{an}  ɓaa   keloh-hii-ri.\\
       grandmother   think-\textsc{pst} \textsc{comp}  individual hear-\textsc{asp.neg-obj.3sg}\\
\glt   ‘Grandmother thought that the person hadn’t heard her.’
\z
\z

In general, complementisers are historically related to demonstratives, but it seems that this relationship is usually mediated by the use of demonstratives in relative and quotative constructions. In particular, the extension of demonstrative relative markers to demonstrative complementisers is cross-linguistically very common \citep[248-254]{SchmidtkeBode2014}. 

\section{Conjunctive adverbs}\label{sec:diessel:6}

Conjunctive adverbs are paratactic clause linkers that combine two independent sentences. In contrast to many other types of clause linkers, they have received little attention in typology. In studies of English grammar, the term conjunctive adverb applies to discourse connectives such as \textit{however,} \textit{thus} and \textit{nevertheless}. Similar types of discourse connectives occur in many other languages and often involve demonstratives. In what follows, we provide an overview of the conjunctive adverbs in our database concentrating on those forms that involve demonstratives. As we will see, conjunctive adverbs vary along several dimensions:

\begin{enumerate}
\item They can be more or less complex ranging from mono-morphemic words to (frozen) multi-word expressions.
\item They are usually associated with the second conjunct but exhibit different degrees of formal integration.
\item They express a wide range of semantic relations including, above all, relations of time, cause and reason.
\end{enumerate}

In some languages, conjunctive adverbs are based on manner demonstratives. In English, for example, the manner demonstratives \textit{so} and \textit{thus} are commonly used as conjunctive adverbs that designate a consequence or logical conclusion \REF{ex:diessel:25}. Likewise, Finnish \textit{niin} ‘so/thus’ and Japanese \textit{koo/so/aa} ‘in this/that way’ are manner demonstratives that can be used as conjunctive adverbs \citep{König2012}.

\ea\label{ex:diessel:25}
{He failed the exam; \textbf{thus/so}, he had to repeat the class.}\\
\z

Apart from manner demonstratives, oblique demonstrative pronouns provide a common source for conjunctive adverbs. In Yurakaré, for example, temporal clauses are introduced by \textit{latijsha}, which is composed of three morphemes: the endophoric reference marker \textit{l-}, the anaphoric medial demonstrative \textit{ati} and the ablative case marker \textit{=jsha} \REF{ex:diessel:26}.

\ea\label{ex:diessel:26}
\langinfo{Yurakaré}{Isolate}{\citealt[321]{VanGijn2006}}\\
\gll mi-bëjti    së=ja       \textbf{latijsha}   shuyuj-ta-m.\\
     \textsc{2sg-}see\textsc{.1sg.sbj}  \textsc{1sg.prn=emph}  then    hidden\textsc{-mid-2sg.sbj}\\
\glt ‘I saw you, then you hid yourself.’
\z

Santali also uses oblique demonstratives to indicate sequential links between two structurally independent sentences. Result and causal clauses are introduced by \textit{ɛnte} ‘because/for’ or \textit{onate} ‘therefore/so.that’, which are based on the demonstratives \textit{ɛn} ‘that’ and \textit{ona} ‘that.\textsc{inan}’ and the instrumental suffix \textit{-te} \REF{ex:diessel:27}.

\ea\label{ex:diessel:27}
\langinfo{Santali}{Austro-Asiatic, Munda}{\citealt[180]{Neukom2001}}\\
\gll am-ṭhɛn-ge    baba-ɲ     cala-k’-kan-a; {\ob}\textbf{ɛnte}=ɲ    baḍae-y-et’-a …{\cb}.\\
     \textsc{2sg-dat-foc}    father-\textsc{1sg.sbj}   go-\textsc{mid-ipfv-ind}   {\db}that\textsc{.ins=1sg.sbj}   know-\textsc{ep-ipfv-ind} \\
\glt ‘I am coming to you, father, because I know …’
\z

Functionally equivalent to oblique demonstratives are adpositional constructions consisting of a demonstrative pronoun or adverb and an adposition. English \textit{therefore}, for instance, derives from Old English \textit{þærƒore} ‘for that’, which is composed of the demonstrative \textit{þær} ‘there’ and the adposition \textit{fore} ‘before, because of’. Similar types of conjunctive adverbs occur in many other languages of our sample. Some examples are given in \tabref{tab:diessel:2}.

\begin{table}
\begin{tabularx}{\textwidth}{lXll}
\lsptoprule
\textbf{Language} & \textbf{Form} & \textbf{Gloss} & \textbf{Translation}\\
\midrule
German & \textit{darum} \newline (< \textit{daːr-umbi}) & that\textsc{.obl}-because.of & ‘therefore’\\
Japanese & \textit{sore-kara}  & that-after & ‘and then’\\
Burmese & \textit{da=jaun} & this=because.of & ‘therefore’\\
Awa Pit & \textit{suna=akwa} & that=because.of & ‘because of that’\\
Supyire & \textit{lire e} & this in & ‘so, therefore’\\
Epena Pedee & \textit{maa-p\textsuperscript{h}éda} & like.that-after& ‘after that’\\
Menya & \textit{i-ta-ŋi} & that-from-given & ‘after that, as a result’\\
Hixkaryana & \textit{ɨre ke} & that because.of & ‘therefore’\\
Koyra Chiini & \textit{woo di banda} & \textsc{dem} \textsc{def} behind  & ‘afterwards’\\
Chumash  & \textit{ʔakim-pi}   & there-\textsc{loc}  & ‘during (that time)’\\
\lspbottomrule
\end{tabularx}
\caption{Examples of conjunctive adverbs}
\label{tab:diessel:2}
\end{table}

Conjunctive adverbs of this type are commonly used to indicate relations of cause, reason and time. Some of these expressions may still be seen as adpositional constructions, but others have turned into monomorphemic clause linkers. At the initial stage of the development, the demonstrative directs the interlocutors’ attention to a previous clause or proposition and the adposition specifies a particular semantic relationship between the two clauses. Yet, as the development continues, the demonstrative and the adposition may lose their status as independent words and may fuse into a single morpheme (e.g. German \textit{darum}).

There are also some languages in our sample in which conjunctive adverbs are based on demonstratives and topic or focus markers. In Galo (Sino-Tibetan), for instance, sequential relations of time and result are expressed by \textit{okkəə} ‘and, then, so’, which derives from the ablative demonstrative \textit{okə} ‘this.\textsc{near.you}’ and the topic marker \textit{əə} \citep[370]{Post2007}. Similarly, in Bilua some “linking adverbs” are based on a distal demonstrative and a focus marker \REF{ex:diessel:28}.

\ea\label{ex:diessel:28}
\langinfo{Bilua}{Solomons East Papuan}{\citealt[45]{Obata2003}}\\
\ea  \textit{sainio} ‘therefore, then’ < \textit{sai} \textit{inio} ‘there \textsc{foc’}\\
\ex  \textit{soinio} ‘therefore, accordingly’ < \textit{so} \textit{inio} ‘that \textsc{foc’}\\
\z
\z

Apart from manner demonstratives and adpositional phrases, linking clauses provide a common source for conjunctive adverbs. There are various types of linking clauses (cf. \citealt{Guérin2019}), but many of them are organised around a demonstrative and a proverb verb such as ‘be’ or ‘do’, as in \REF{ex:diessel:29} and \REF{ex:diessel:30} from Alamblak and Manambu.

\ea\label{ex:diessel:29}
\langinfo{Alamblak}{Sepik, Sepik Hill}{\citealt[283]{Bruce1984}}\\
\gll yira   buga-m   fa-më-r-m.     {\ob}\textbf{ɨnd-net-r-n},   yati-fa-më-r.{\cb}\\
     fish   all-3\textsc{pl}   eat-\textsc{rpst}-3\textsc{sg.m}-3\textsc{pl}   {\db}\textsc{dem}-do-3\textsc{sg.m}-\textsc{dep} stomach-eat-\textsc{rpst-3sg.m}\\
\glt ‘He ate all the fish. He did that (therefore), he had a stomach ache.’
\z

\ea\label{ex:diessel:30}
\langinfo{Manambu}{Sepik, Middle Sepik}{\citealt[494]{Aikhenvald2008}}\\
\gll sanaːk   karabə    jaːp  kui-taka-dana-ti, {\ob}\textbf{alək}    \textbf{tə-ku}    maː  tə{\cb}.\\ 
money.\textsc{lk.dat}   men’s.house.\textsc{lk}  thing give.to.third.\textsc{pst}-put-\textsc{3pl.sbj-3pl} {\db}that.\textsc{dat} \textsc{be-asp.ss}  \textsc{neg} \textsc{have.neg}\\
\glt ‘They gave away the things from men’s house for money, this is why they do not have (them anymore).’
\z

In both examples, the second sentence is connected to the preceding sentence by a (linking) clause that includes a demonstrative pronoun and a proverb. In Alamblak, the linking clause is a single word consisting of the demonstrative \textit{ɨnd}, the proverb \textit{net} ‘do’, a third person suffix and a dependent marker. In Manambu, the linking clause is composed of the distal demonstrative \textit{a} ‘that’ (in dative case) and a medial clause including the verb \textit{tə} ‘be, stand’.\footnote{Medial clauses are dependent clauses of clause chaining constructions. They occur with switch-reference markers that indicate whether the subsequent clause includes the same or a different subject \citep{HaimanMunro1983}.} Note that both the demonstrative and the medial clause are also used alone for clause combining, but according to \citet[494]{Aikhenvald2008} the expression \textit{alək tə-ku} is in the process of developing into a complex clause linker meaning ‘and so, as a result’. 

Linking clauses of this type provide a common strategy of clause combining in Papuan languages (e.g. Alamblak, Manambu, Korafe, Menya) but also occur in other languages in our sample. Korean, for instance, has a whole series of “conjunctive adverbials” that derive from linking clauses including the demonstrative \textit{ku} ‘that’ and the verb \textit{ha(y)} ‘do, be’ \REF{ex:diessel:31}.

\ea\label{ex:diessel:31}
\langinfo{Korean}{Isolate}{\citealt[89-90]{Sohn1994}, \citeyear[292]{Sohn2009}}\\
\ea  \textit{kulayse} ‘so, thus, therefore’ < \textbf{\textit{ku}}\textit{-li/le-hay-se} ‘that-along/like-do/be-as’\\
\ex \textit{kuliko} ‘and’ < \textbf{\textit{ku}}\textit{-li-ha-ko} ‘that-along/like-do-and’\\
\ex  \textit{kulehciman} ‘but, however’ < \textbf{\textit{ku}}\textit{-li/le-ha-ciman} ‘that-along/like-do/be-though’\\
\ex  \textit{kulinikka} ‘therefore’ < \textbf{\textit{ku}}\textit{-li/le-ha-nikka} ‘that-along/like-do/be-because’\\
\z
\z

\section{Adverbial subordinate conjunctions}\label{sec:diessel:7}

Adverbial clauses are subordinate clauses that express a wide range of semantic relations (\citealt{ThompsonEtAl2007}; see also \citealt{Diessel2019Network}). Since many of these relations are also expressed by adpositional phrases, it is not surprising that adverbial clauses are often marked by adpositions. In English, for example, some temporal adverbial clauses are introduced by subordinate conjunctions that are also used as temporal prepositions (e.g. \textit{since}, \textit{after}, \textit{before}). 

Across languages, there is a close connection between adverbial subordinators and certain semantic types of adpositions, notably adpositions of time, cause and purpose. However, in addition to adpositions, adverbial clauses occur with a wide range of other subordinating morphemes, including morphemes that are historically related to demonstratives. In German, for example, some adverbial clauses of time and purpose include the demonstratives \textit{dem} ‘that.\textsc{dat}’ \REF{ex:diessel:32a}, \textit{da} ‘there’ \REF{ex:diessel:32b} and \textit{so} ‘so, thus’ \REF{ex:diessel:32c}.

\ea\label{ex:diessel:32}
{Modern German (Indo-European, Germanic)}\\
\ea\label{ex:diessel:32a} \textit{seit}\textbf{\textit{dem}} ‘since’, \textit{nach}\textbf{\textit{dem}} ‘after’, \textit{in}\textbf{\textit{dem}} ‘by’\\
\ex\label{ex:diessel:32b} \textbf{\textit{da}}\textit{mit} ‘in order to, so that’, \textbf{\textit{da}} ‘since, as, because’\\
\ex\label{ex:diessel:32c} \textbf{\textit{so}}\textit{bald} ‘as soon as’, \textbf{\textit{so}}\textit{fern} ‘as long as’\\
\z
\z

Note that some of the subordinate conjunctions in \REF{ex:diessel:32} are composed of a demonstrative and an adposition, similar to conjunctive adverbs such as \textit{darum} ‘therefore’ (see \sectref{sec:diessel:6}). We will come back to this below. Here we note that while subordinate conjunctions are often similar to conjunctive adverbs, there is a clear structural difference between them in Modern German. In contrast to conjunctive adverbs (e.g. \textit{darum}), adverbial subordinators (e.g. \textit{damit}) introduce subordinate clauses that are distinguished from main clauses, or paratactic sentences, by a particular word order. As can be seen in \REF{ex:diessel:33a}, in adverbial clauses, the finite verb occurs in clause-final position, whereas in main clauses \REF{ex:diessel:33b}, the finite verb comes in second position, i.e. right after the conjunctive adverb. Thus, while it is often said that adverbial clauses and paratactic sentences from a continuum (e.g. \citealt[237]{ThompsonEtAl2007}), there are languages like German in which the continuum is divided into separate constructions.

\ea\label{ex:diessel:33}
{Modern German (Indo-European, Germanic)}\\
\ea\label{ex:diessel:33a}
\gll   Wir   gehen  jetzt,   \textbf{damit}   wir   nicht   zu   spät   sind.\\
       we   go  now  so.that  we  not  too  late  are\\
\glt   ‘We are leaving now so that we won’t be late.’

\ex\label{ex:diessel:33b}
\gll   Wir   haben   den   Zug   verpasst; \textbf{darum}   sind   wir   zu  spät.\\
       we   have  the  train  missed   therefore  are  we  too  late\\
\glt   ‘We missed the train; that’s why we are too late.’
\z
\z

Apart from German, there are several other languages in our sample in which adverbial clauses are marked by subordinate conjunctions that are etymologically related to demonstratives. For example, in English, result clauses are introduced by \textit{so} \textit{that}, and in French, conditional clauses are introduced by the manner demonstrative \textit{si} ‘if’ \citep{König2012}. Two further examples of demonstrative subordinate conjunctions are shown in \REF{ex:diessel:34} and \REF{ex:diessel:35}.

\ea\label{ex:diessel:34}
\langinfo{Tamashek}{Afro-Asiatic, Berber}{\citealt[663]{Heath2005}}\\
\gll {\ob}\textbf{\`ɑ}=s   Ø-æmmu-t{\cb}     n-\`əglɑ.\\
     {\db}\textsc{dem=ins}   \textsc{3m.sg.sbj.}die.\textsc{asp-aug}   \textsc{1sg.sbj-}go.away.\textsc{asp}\\
\glt ‘When he died, we went away.’
\z

\ea\label{ex:diessel:35}
\langinfo{Yimas}{Lower Sepik-Ramu, Lower Sepik}{\citealt[453]{Foley1991}}\\
\gll {\ob}\textbf{m}-n-awram-r-mp-n{\cb}     mpa-n  namarawt  anak.\\
     {\db}\textsc{dem}-\textsc{3sg.a}-enter-\textsc{pfv-sg-obl}   one-\textsc{sg}  person.\textsc{sg}  \textsc{cop.1sg}\\
\glt ‘When he went in, he went alone.’
\z

As can be seen, in Tamashek temporal ‘when’ clauses are marked by a demonstrative and an instrumental case clitic \citep[663]{Heath2005}, and in Yimas “finite oblique clauses”, which are functionally equivalent to adverbial clauses in English, are expressed by nominalisations that begin with the “near distal deictic base” \textit{m-} ‘that’ \citep[435]{Foley1991}. Other languages in which some adverbial subordinate conjunctions are historically related to demonstratives include Wariˈ (time clauses), Jamul Tiipay (time and purpose clauses) and Bilinarra (conditional clauses) (see \citealt[250-251]{HeineKuteva2007} for additional examples).

Given that the subordinate conjunctions of adverbial clauses are often similar to conjunctive adverbs (e.g. German \textit{damit} ‘so that’ with \textit{darum} ‘therefore’), we may hypothesise that (some) subordinate conjunctions derive from paratactic clause linkers \citep[185]{HopperTraugott2003}. However, while this hypothesis is not implausible, there is little evidence for it in our data. On the contrary, the available data suggest that the demonstratives of adverbial subordinators do not usually derive from paratactic clause linkers but from demonstratives of other types of subordinate clauses. In German, for instance, adverbial conjunctions such as \textit{seit\textbf{dem}} ‘since’ and \textit{nach\textbf{dem}} ‘after’ are not derived from conjunctive adverbs of paratactic sentences but from oblique relative clauses in Old and Middle High German, e.g. \textit{\textbf{sît dem} mâle daʒ} ‘since the time that’ \citep[238]{Lockwood1968}. Similar types of adverbial subordinators occur in Tamashek \citep{Heath2005} and Yimas \citep{Foley1991}, in which adverbial clauses are marked by demonstratives that also occur in relative clauses. While there are no diachronic data to investigate the diachronic origins of adverbial subordinators in Tamashek and Yimas, \citet[663-675]{Heath2005} and \citet[435-444]{Foley1991} make it clear that the adverbial clauses of these languages are derived from oblique relatives. 

More research is needed to determine the diachronic trajectories of demonstrative subordinate conjunctions in adverbial clauses, but judging from the evidence in our database we suspect that the demonstratives of adverbial subordinate conjunctions are more frequently derived from demonstrative relativisers, complementisers or linking and nominalising articles than from discourse deictic demonstratives or paratactic clause linkers. 

\section{Correlatives}\label{sec:diessel:8}

The notion of correlative is used in many different ways in linguistics \citep{Lipták2009}. In the current study, we use the term correlative for pronominal and conjunctive elements of main clauses that serve to indicate the occurrence of an associated subordinate clause (or a particular element within the subordinate clause). Since subordinate clauses are commonly marked by a subordinating morpheme – a relativiser, complementiser or adverbial conjunction – a correlative is often used together with a subordinate marker. In conditional sentences, for instance, subordinate conjunctions are often paired with a correlative in the main clause (e.g. English \textit{if/then}).

What is important in the context of the current chapter is that correlatives are very often based on demonstratives. Consider, for instance, the two following examples of conditional sentences from German \REF{ex:diessel:36} and Hungarian \REF{ex:diessel:37}.

\ea\label{ex:diessel:36}
{Modern German (Indo-European, Germanic)}\\
\gll Auch   wenn   noch   vieles  unklar   ist,  {\op}\textbf{so}{\cp}  müssen wir  doch  jetzt  handeln. \\
     Even   though  still  much  unclear  is  {\db}so  must we  still  now  act\\
\glt ‘Even though much is still unclear, we must act now.’
\z

\ea\label{ex:diessel:37}
\langinfo{Hungarian}{Uralic, Ugric}{\citealt[51]{KeneseiEtAl1998}}\\
\gll Ha   Péter   el-alszik,   {\op}\textbf{akkor}{\cp}   Anna   meg-haragszik.\\
     if   Peter   \textsc{pre}-sleeps   {\db}then   Anna   \textsc{pre}-is.angry\\
\glt ‘If Peter falls asleep, Anna will get angry.’
\z

In both languages the main clauses of conditional sentences are optionally introduced by a correlative. The German correlative \textit{so} is a manner demonstrative that is also used in many other contexts (cf. \sectref{sec:diessel:4} to \sectref{sec:diessel:6}),\footnote{Interestingly, \textit{so} is not only used as a correlative, it can also function as a conditional conjunction, similar to French \textit{si} ‘if’ (< \textit{sīc} ‘thus, so’) (e.g. \textit{\textbf{So} Gott will, wird er wieder gesund} ‘If God wants (it), he will get well’). According to \citet{Traugott1985}, conditional \textit{so} and \textit{si} were originally used as correlatives that were later extended to subordinate clauses and reanalysed as conditional conjunctions (see also \citealt{Harris1986}).} and the Hungarian correlative \textit{akkor} ‘then’ is composed of the demonstrative \textit{az} ‘that’ and the temporal suffix \textit{-kor} ‘at (the time)’.

Correlatives are not only used with conditional clauses; they also occur with other semantic types of adverbial clauses. The following examples from German, Hungarian and Georgian include demonstrative correlatives that serve to anticipate upcoming adverbial clauses of manner \REF{ex:diessel:38}, cause \REF{ex:diessel:39} and result \REF{ex:diessel:40}.

\ea\label{ex:diessel:38}
{Modern German (Indo-European, Germanic)}\\
\gll Ich  mache  das   \textbf{so},  wie  du  gesagt  hast.\\
     I  do  that   so  as  you  said  have\\
\glt ‘I will do it (in the way) as you said.’
\z

\ea\label{ex:diessel:39}
\langinfo{Hungarian}{Uralic, Ugric}{\citealt[51]{KeneseiEtAl1998}}\\
\gll Anna   \textbf{az-ért}  haragszik,   mert    Péter  elaludt.\\
     Anna   that-for  is.angry      because   Peter  slept\\
\glt ‘Anna is angry because Peter has fallen asleep.’
\z

\ea\label{ex:diessel:40}
\langinfo{Georgian}{Kartvelian}{\citealt[578]{Hewitt1995}}\\
\gll Ik    \textbf{iset}-i      mgl-eb-i    da   t’ur-eb-i ar-i-an,\\
     there \textbf{like.that}-\textsc{agr}  wolf-\textsc{pl-nom}   and   jackal-\textsc{pl-nom} be-\textsc{prs}-they\\
\gll rom   še-g-č’am-en.\\
     \textsc{sub}   \textsc{pre-}you-devour-they.\textsc{fut}\\
\glt ‘There are such wolves and jackals there that they will devour you.’
\z

Like adverbial clauses, relative clauses may occur with a correlative in the main clause. Linguistic typologists distinguish between several types of relative constructions and one of them is the correlative relative clause \citep{Lipták2009}. Correlative relatives were very frequent in the ancient Indo-European languages (e.g. Hittite, Sanskrit) and are still the dominant relative construction in the Indic branch of modern Indo-European languages \citep{Srivastav1991}. In Hindi, for example, the most frequent type of relative clause is a correlative construction in which the relative clause typically precedes the main clause as in \REF{ex:diessel:41}.

\ea\label{ex:diessel:41}
\langinfo{Hindi}{Indo-European, Indic}{\citealt[1]{Lipták2009}}\\
\gll {\ob}jo  laRkii  khaRii hai{\cb} \textbf{vo}  lambii  hai.\\
     {\db}\textsc{rel}    girl  standing is that  tall  is\\
\glt ‘The girl who is standing is tall.’
\z

The relative clauses of correlative constructions are non-embedded clauses that typically include the head they modify. In \REF{ex:diessel:41}, \textit{laRkii} ‘girl’ is the nominal head of the relative clause, which is marked by the morpheme \textit{jo} and resumed in the second clause by the correlative \textit{vo} ‘that’. \textit{Vo} is a case-inflected demonstrative pronoun that is obligatory in this context and serves to indicate the syntactic function of the head within the main clause.

Similar types of correlative relative clauses occur in other languages of our sample. Like the correlative constructions of Hindi (and other Indic languages), the correlative constructions of these languages consist of non-embedded relative clauses in which the nominal head is “represented” by a demonstrative correlative in the main clause. Two examples from Wappo and Georgian are given in \REF{ex:diessel:42} and \REF{ex:diessel:43}.

\ea\label{ex:diessel:42}
\langinfo{Wappo}{Wappo-Yukian}{\citealt[115]{ThompsonEtAl2006}}\\
\gll ah     {\ob}i-ø   k'ew-ø  naw-ta{\cb}   \textbf{ce}    hak'-šeʔ.\\
     \textsc{1sg.nom}   {\db}\textsc{1sg-acc}   man-\textsc{acc}   see-\textsc{pst.dep}   \textsc{dem} like-\textsc{dur}\\
\glt ‘I like the man I saw.’
\z

\ea\label{ex:diessel:43}
\langinfo{Georgian}{Kartvelian}{\citealt[607]{Hewitt1995}}\\
\gll {\ob}gušin  rom beč’ed-i   {\op}Ø-Ø-{\cp}m-a-čuk-e{\cb}, \textbf{is}     {\op}beč’ed-i{\cp}    sad  ar-i-s?\\
     {\db}yesterday  \textsc{sub}  ring-\textsc{nom} {\db}you-it-me-\textsc{loc}-present-\textsc{aor.ind} that.\textsc{nom}   {\db}ring-\textsc{nom}  where  be-\textsc{prs}-it\\
\glt ‘Where is that ring which you presented to me yesterday?’
\z

Note that the correlative relative clauses in Wappo do not include a marker of the head noun (parallel to Hindi \textit{yo}) and that the correlative constructions in Georgian may include a copy of the head in the second clause (i.e. \textit{beč’ed-i} ‘ring-\textsc{nom}’). In general, correlative relative constructions are very flexible. There is a tendency to prepose the relative clause, but in all of the languages with correlative relatives in our sample, the relative clause may also be postposed to the main clause, and the head noun may occur either within the relative clause (which is most frequent) or in the main clause or in both clauses.

Finally, there are also some languages in our sample in which complement clauses occur with a correlative pronoun. Depending on the order of main and complement clause, the correlative is either forward referring, as in \REF{ex:diessel:44} from Hungarian, or it is backwards referring, as in \REF{ex:diessel:45} from Thai Kamti. Note that \citet[119]{Inglis2014} refers to the demonstrative in \REF{ex:diessel:45} as a “complement marker”, but given that \textit{nai\textsuperscript{1}} ‘this’ serves as object of the second clause, we consider \textit{nai\textsuperscript{1}} a backwards referring correlative rather than a complementiser.

\ea\label{ex:diessel:44}
\langinfo{Hungarian}{Uralic, Ugric}{\citealt[28]{KeneseiEtAl1998}}\\
\gll Anna tudta (\textbf{azt}), hogy Péter beteg.\\
     Anna knew.\textsc{def} that/it.\textsc{acc} that Peter sick \\
\glt ‘Anna knew that Peter was sick.’
\z

\ea\label{ex:diessel:45}
\langinfo{Thai Kamti}{Tai-Kadai; Kam-Tai}{\citealt[119]{Inglis2014}}\\
\gll {\ob}tang\textsuperscript{4}  man\textsuperscript{4}  uu\textsuperscript{5}{\cb}  \textbf{nai\textsuperscript{1}}    kau\textsuperscript{3}  piuu\textsuperscript{5}  uu\textsuperscript{5}.\\
     {\db}with   \textsc{3sg}  live  \textsc{comp}   \textsc{1sg}  be.happy  \textsc{impf}\\
\glt ‘(I) am happy that (I) live with her.’ (Lit. ‘I live with her, this I am happy (about).’)
\z

\section{Topic markers}\label{sec:diessel:9}

The final type of clause linker to be considered in this chapter serves to mark topics. In their basic use, topic markers accompany nominal constituents, but in some languages topic markers also occur with subordinate clauses. For instance, in many languages conditional clauses include a topic marker \citep{Haiman1985}. \citet{Haiman1978} argued that the frequent use of topic markers in conditional clauses is motivated by the communicative function of conditionals to lay the foundation for the interpretation of subsequent clauses (see also \citealt{Diessel2005Demonstratives}). But topic markers do not only occur in conditionals; they also appear in other types of preposed adverbial clauses \citep{ThompsonEtAl2007} and certain types of relative clauses \citep{Vries1995}. 

Topic markers are often historically related to copulas and adpositions (e.g. \textit{as} \textit{for}), but also develop from demonstratives. There are, for instance, several Papuan languages in our sample in which noun phrases, preposed adverbial clauses and internally headed relatives occur with the same demonstrative as topic marker, as in \REF{ex:diessel:46} from Usan.

\ea\label{ex:diessel:46}
\langinfo{Usan}{Trans-New Guinea, Madang}{\citealt[182, 200, 187]{Reesink1984}}\\
\ea \gll   {\ob}munai  âib \textbf{eng}{\cb}    yonou  bain     mindat-erei.\\
       {\db}house  big this.\textsc{given}  my  older.brother   build-3\textsc{sg.pst}\\
\glt   ‘The big house, my older brother built.’
\ex
\gll {\ob}wau   eâb   igor-iner  \textbf{eng}{\cb}   unor   mâni   utibâ.\\
    {\db}child   cry.\textsc{ss}   be-\textsc{3sg.fut}  this.\textsc{given} mother yam she.will.give.him\\
\glt   ‘If the child is crying, his mother will give him yam.’
\ex
\gll   {\ob}qemi  eng  munon   bau-or   \textbf{eng}{\cb}  ye   me   ge-au.\\
       {\db}bow  this.\textsc{given}  man   take-\textsc{3sg.pst}  this.\textsc{given}  I   not   see-\textsc{nom}  \\
\glt   ‘The bow that the man took I did not see.’
\z
\z

In all three examples, the initial constituent is marked by \textit{eng} ‘this.\textsc{given’,} which is composed of the proximal demonstrative \textit{e} ‘here, this’ and a marker for given information. If \textit{eng} is used as a topic marker, it follows the associated noun phrase or subordinate clause, but \textit{eng} can also function as an independent pronoun meaning ‘this/that one’ \citep[209]{Himmelmann1997}.

Similar types of topic markers occur in several other Papuan languages (Wambon, Korafe, Menya, Urim). For instance, Wambon \REF{ex:diessel:47} and Korafe \REF{ex:diessel:48} use demonstratives at the end of preposed subordinate clauses that one might analyse as topics.

\ea\label{ex:diessel:47}
\langinfo{Wambon}{Trans-New Guinea, Awju-Dumut}{\citealt[518]{Vries1995}}\\
\ea \gll   {\ob}Wano-ne-e  moke-knde-n-\textbf{eve}{\cb}  kaimo-nde  koyomke-khe.  \\
       {\db}child-\textsc{trs-conn}  be.afraid-3\textsc{pl.prs-tr}-that  teacher\textsc{-conn}  be.angry\textsc{-3sg.prs}\\
\glt   ‘The children are afraid because the teacher is angry.’
\ex
\gll   {\ob}Alive   ndu-ne-e  takhima-lepo-n-\textbf{eve}{\cb}  kaimo-nde.\\
{\db}yesterday sago-\textsc{tr-conn}  {buy}\textsc{-1sg.pst-tr}-\textsc{dem} {good-is}\\
\glt   ‘The sago which I bought yesterday, is good.’
\z
\z

\ea\label{ex:diessel:48}
\langinfo{Korafe}{Trans-New Guinea, Binanderean}{\citealt[77, 78]{Farr1999}}\\
\ea \gll   {\ob}Nande   mandi   evetu-fifitu-sira   \textbf{a=mo}{\cb},  jo   taima=da  sumb-ae=ri.\\
       {\db}\textsc{1sg.gen}   boy   woman-put-\textsc{pst.3sg}   that-\textsc{top}  \textsc{neg}   bush=\textsc{loc}  run-not.do=\textsc{cop.q}\\
\glt   ‘When my son got married, he didn’t run away with her into the bush.’
\ex
\gll   {\ob}Nande   mandi   evetu-fifitu-sira   \textbf{a=mo}{\cb},   oroko  Moresby  ir-ira.\\
       {\db}\textsc{1sg.gen}   boy   woman-put-\textsc{pst.3sg} that-\textsc{top}  today  Moresby remain-\textsc{prs.3sg}\\
\glt ‘My son that’s married is living in Moresby now.’
\z
\z

On the face of it, the subordinate clauses in these examples look similar to some of the nominalised clauses that we have seen in \sectref{sec:diessel:3}. In particular, Jamul Tiipay and Assiniboine have relative and complement clauses that end with a demonstrative (cf. \REF{ex:diessel:8c} and \REF{ex:diessel:9c}), but in contrast to the clause-final demonstratives of Usan, Wambon and Korafe, the clause-final demonstratives of Jamul Tiipay and Assiniboine do not occur in conditional clauses and do not seem to serve as topic markers (according to our sources).

Since there are no diachronic corpora to study the development of demonstrative topicalisers, we cannot be certain how these markers have evolved. Yet, \citet{Reesink1984} and \citet{Vries1995} proposed a scenario which, we believe, provides a plausible account for their development. Both scholars observe that topicalised subordinate clauses in Papuan languages are (often) resumed by a correlate pronoun at the beginning of the main clause, as in \REF{ex:diessel:49} from Wambon.

\ea\label{ex:diessel:49}
\langinfo{Wambon}{Trans-New Guinea, Awju-Dumut}{\citealt[517, 518]{Vries1995}}\\
\ea \gll   {\ob}Ko  mba-khe-n-o    kav=eve{\cb}  \textbf{eve}   na-mbap-nde.\\
       {\db}there  stay-\textsc{3sg.prs-tr-conn}  man=that  that   my-father-is\\
\glt ‘The man who is staying there, that is my father.’
\ex
\gll   {\ob}Kikhuve   ndethekhel=eve{\cb}     \textbf{eve}  Manggelum  konoksiva.    \\
       {\db}Digul     rise.3\textsc{sg.cond}=that   that  Manggelum   go\textsc{.neg.1pl.intent}\\
\glt ‘If the Digul river rises, then we do not want to go to Manggelum.’
\z
\z

According to \citeauthor{Vries1995}, the demonstrative topicalisers of Wambon (and other Papuan languages) are derived from demonstrative correlatives that have become associated with the preceding subordinate clause. There is good evidence for this hypothesis, especially in Wambon. Since \textit{eve} ‘that’ is a demonstrative pronoun that cannot be interpreted as a determiner if it follows an NP or clause (demonstrative determiners precede NPs in Wambon), it seems reasonable to assume that \textit{eve} evolved from a correlate pronoun rather than from a nominalising article, or determiner, as some of the demonstrative clause linkers described in \sectref{sec:diessel:3} \citep[187-188]{Reesink1984}.

\section{Summary and conclusion}

To conclude, demonstratives are of fundamental importance to clause combining. They are commonly used as anaphors and discourse deictics and provide a very frequent source for the development of various types of grammatical clause linkers. Some of these developments are frequently mentioned in textbooks and handbook chapters on grammaticalisation, but others have only been described in reference grammars and other special sources. The current study provides the first large-scale investigation of demonstrative clause linkers from a cross-linguistic perspective. Drawing on data from a sample of 100 languages, the chapter has analysed eight basic types of clause linkers that are frequently derived from a demonstrative:

\begin{enumerate}
\item
Relative pronouns
\item
Linking and nominalising articles
\item
Quotative markers
\item
Complementisers
\item
Conjunctive adverbs
\item
Adverbial subordinate conjunctions
\item
Correlatives
\item
Topic markers
\end{enumerate}

There is abundant evidence in our database that all of these markers are often etymologically related to demonstratives. Yet, while the deictic origins of many clause linkers are morphologically transparent, it is not always clear how they evolved. In accordance with the current literature on grammaticalisation, we have argued that the development of demonstratives into grammatical clause linkers is crucially influenced by the constructions in which demonstratives occur. If we want to understand how and why demonstratives develop into grammatical clause linkers, we must not only consider the discourse-pragmatic uses of demonstratives but also their syntactic functions.

One aspect that is not always recognised in the literature on grammaticalisation is that not all demonstrative clause linkers are immediately derived from demonstrative anaphors and discourse deictics. As we have seen, the various types of demonstrative clause linkers are historically related to each other and these relationships are crucial to understand the occurrence of demonstratives in certain clause-linkage constructions. In particular, the analysis of demonstratives in subordinate clauses needs to take into account that the various types of subordinate markers are historically related \citep{SchmidtkeBode2014,Diessel2019Network}. For instance, contrary to what is commonly assumed in the literature (e.g. \citealt[184-185]{HopperTraugott2003}), the demonstratives of nominal and adverbial clauses are often based on demonstrative relative markers and articles rather than on demonstrative clause linkers of paratactic sentences. While there are languages in which complementisers and adverbial subordinators are immediately derived from the discourse uses of demonstratives (see \citealt{McConvell2006} for some examples from Australian languages), this does not seem to be a major path of evolution. 

Finally, on a more general note, this chapter presents new evidence for \citegen{Brugmann1904} and \citegen{Bühler1934} claim that many grammatical function morphemes have a deictic origin. Current research on grammaticalisation has been mainly concerned with the development of grammatical markers from content words and has paid little attention to demonstratives. In fact, some researchers have argued that all grammaticalisation processes evolve from nouns and verbs \citep[111]{HeineKuteva2007}. However, in addition to nouns and verbs, demonstratives provide an extremely frequent source for the development of a wide range of grammatical markers, including the many different types of clause linkers investigated in the current study. More research is needed to better understand the mechanisms behind some of the developments considered in this chapter. Yet, there is no doubt that demonstratives are of fundamental significance to the diachronic evolution of grammar including the evolution grammatical clause linkers and clause linkage constructions.

\section*{Abbreviations}

The chapter abides by the Leipzig Glossing Rules. Additional or deviant abbreviations include:

\begin{tabularx}{.45\textwidth}{lQ}
\textsc{act} & actor voice\\
\textsc{aor} & aorist\\
\textsc{asp} & aspect\\
\textsc{asp.neg} & aspect negation\\
\textsc{aug} & augment\\
\textsc{conn} & connective \\
\textsc{ct} & contrastive\\
\textsc{dp} & dependent marker\\
\textsc{emph} & emphatic\\
\textsc{ep} & epenthetic\\
\textsc{evd} & indirect evidence\\
\textsc{given} & given information\\
\textsc{hab} & habitual\\
\textsc{impf} & imperfect\\
\textsc{inan} & inanimate\\
\textsc{intent} &  intentional\\
\textsc{lk} & linker/linking article\\
\textsc{med} & medial demonstrative\\
\end{tabularx}
\begin{tabularx}{.45\textwidth}{lQ}
\textsc{mid} & middle voice\\
\textsc{narr} & narrative\\
\textsc{nhyp} & nonhypothetical\\
\textsc{out} & V outward\\
\textsc{pat} & patient\\
\textsc{pot} & potential, hypothetical\\
\textsc{pre} & prefix\\
\textsc{pres} & presentative\\
\textsc{prn} & pronoun\\
\textsc{rpst} & remote past tense\\
\textsc{rc} & relative construction\\
\textsc{sjrel} & subject relative\\
\textsc{ss} &  same subject\\
\textsc{st} & first part of a discontinuous root\\
\textsc{start} & inceptive\\
\textsc{sub} & subordinator\\
\textsc{trs} & transitional sound 
\end{tabularx}

\section*{Appendix: Language sample}

\textsc{Africa}: Fongbe, Hausa, Jamsay, Kana, Khwe, Konso, Koyra Chiini, Krongo, Lango, Mayogo, Mbay, Nkore Kiga, Noon, Supyire, Tamashek

\textsc{North} \textsc{and} \textsc{Central} \textsc{America}: Assiniboine, Choctaw, (Barbareño) Chumash, Kiowa, (Chalcatongo) Mixtec, Musqueam, Ojibwe, Purépecha, Rama, Slave, Tepehua, Jamul Tiipay, Tümpisa Shoshone, Tzutujil, Wappo, West Greenlandic

\textsc{South} \textsc{America}: Aguaruna, Awa Pit, Barasano, Cavineña, Epena Pedee, Hixkaryana, Hup, Jarawara, Kwazá, Mapudungun, Matsés, Mekens, Mosetén, (Huallaga) Quechua, Tariana, Trumai, Urarina, Warao, Wariˈ, Yagua, Yurakaré

\textsc{Eurasia}: Abkhaz, Arabic, Basque, Evenki, French, Georgian, German, Hindi, Hungarian, Japanese, Korean, Lezgian, Malayalam, Santali, Serbo-Croatian, Turkish, (Kolyma) Yukaghir

\textsc{South-East} \textsc{Asia} \textsc{and} \textsc{Oceania}: Burmese, Hmong Njua, Begak Ida’an, Toba Batak, Lao, Mandarin Chinese, Meithei, Semelai, Tagalog, Thai Kamti, Tetun, Toqabaqita, Tukang Besi, Vietnamese

\textsc{Australia} \textsc{and} \textsc{New} \textsc{Guinea}: Alamblak, Ambulas, Amele, Bariai, Kayardild, Komnzo, Korafe, Manambu, Mangarayi, Menya, Motuna, Martuthunira, Ungarinjin, Usan, Wambaya, Wambon, Yimas

\sloppy\printbibliography[heading=subbibliography,notkeyword=this]
\end{document}
